\section{Consola}

En el tp anterior se mostró una implementación de la consola. Esta no se atenía
a la abrstración de dispositivos, era más bien un agregado al core del kernel.

Procedí a separar la lógica de la tty a un driver, conformando el manejo del
teclado, y la lectura/escritura en una terminal virtual. Luego se crean los
dispositivos tty0, tty1, ...

Además se registra el handler de interrupción, y se mantiene una consola como
la actual, que representa aquella con foco. Luego todo interrupción del
teclado es ruteada a la misma, y se mantiene el cursor a la posición actual.

La escritura se resuleve igual que en el módulo de vga del kernel, con la
función print\_key(). La lectura del teclado se apoya en el uso de un semáforo,
que empieza en 0 y es signalizado cada vez que se recibe un '\n'. Luego una
tarea que lea stdin se bloqueará hasta que se reciba el centinela.
