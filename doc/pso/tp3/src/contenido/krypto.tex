\section{Programa Krypto}

Para mostrar un poco esta nueva funcionalidad, se crearon 2 aplicaciones:
krypto y memkrypto.

La primera consiste en 3 procesos, uno que lee un archivo y lo escribe en una
pipe, otro que lee esta pipe, lo ecripta y lo reenvia por otra pipe, y por
último un proceso que lee este archivo encriptado y lo reproduce en stdout (o
un archivo/dispositivo en el disco). Esto muestra un poco el manejo de las
pipes y su funcionalidad.

La otra aplicación es memkrypto, en donde también utilizamos las pipes pero
para sincronización (como mutexes). Luego el verdadero paso de datos se hace
por memoria compartida, utilizando como se explicó anteriormente palloc() y
share\_page().
