\section{Introducción}

Para realizar este trabajo me basé en una pequeña implementación de un kernel
que desarrollé como final para Organización del Computador 2. Este a su vez
está basado en uno aún más primitivo que nos dieron en la cursada. La ventaja
de esto es que el template de OS que se usa en PSO es una evolución del de
Orga2. La desventaja, como se verá más adelante, es que portar todo a este
modelo para no apartarme de la cursada terminó consumiendo más tiempo que si
hubiese empezado de cero.

El port consistió en adaptarme a las convenciones de código, que implicó
modificar los nombres de la mayoría de las funciones, e implementar la
funcionalidad que requería el trabajo. El kernel original era muy simple, solo
proveía la funcionalidad necesaria para poder utilizar el File System, que fue
el verdadero objetivo del tp. Por esta razón todavía se encuentra esta
funcionalidad, hay un fs para MinixFS que provee de llamadas como
open/close/read/write/etc. que no explicaré demasiado ya que no hace a los
requerimientos del TP1 de PSO.

Daré una explicación de la resolución de cada punto, sin ahondar demasiado en
el código, sino en la idea y razón del mismo.

Disclamer: algunas secciones están basada en la explicación para la entrega de
este trabajo como final del orga2.
