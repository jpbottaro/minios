\section{Manejo de pantalla (vga)}

El manejo de pantalla se encuentra implementado en el archivo vga.c.

Para facilitar la programación del mismo definí una estructura que imita los
pares color/letra de la vga de video. Luego bastó definir un puntero a arreglo
de estas esctructuras, y apuntarlo a la dirección 0xB8000 (el arreglo se llama
vram).

La función más importante de este módulo es print\_key(). Dado un caracter en
ascii, lo introduce en la pantalla segun la pocisión del cursor. Además
interpreta una serie de caracteres especiales como el backspace y reproduce su
funcionalidad. Entre estos caracteres se encuentra el de nueva linea, que
únicamente modifica el cursor sin imprimir nada.

En el caso en que el cursor llegue al final de la pantalla y reciba otra nueva
linea, aparece la función scroll\_up\_vram(), que copia el contenido de la
memoria de video una fila hacia arriba para hacer lugar a los nuevos
caracteres.

Por último tenemos real\_printf(), una función que imita a printf() de la
biblioteca estándar de C. Simplemente parsea caracter por caracter el string de
formato e imprime en la pantalla acorde a lo que recibe, llamando a
print\_key(). Hace uso de funciones genéricas que pueden encontrarse en la
carpeta lib/ (clones de funciones de la biblioteca de C).

El resto de las funciones son en su mayoría wrappers de las ya descriptas.
