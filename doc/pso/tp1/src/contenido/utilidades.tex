\section{Aplicaciones}

Tome todas las medidas necesarias para que este kernel sea compatible con C,
con la interfaz POSIX y GCC. Más aún mi objetivo desde un principio fue que las
aplicaciones programadas para este núcleo funcionen sin cambios en linux.
Salvando las distancias, y reduciendose a utilizar únicamente el set pequeño
de funciones que nos da el sistema, proveo una serie aplicaciones que cualquier
usuario de un sistema GNU/Linux va a conocer de inmediato (todas se encuentran
en el directorio apps):

- cash: un "clon" de sh, una consola que recibe comandos y los ejecuta.

- cp: copia un archivo

- echo: imprime una linea en stdout

- ls: lista las entradas de un directorio

- rm: remueve un archivo

- mv: mueve un archivo

- mkdir: crea un directorio

- rmdir: remueve un directorio

- cat: imprime en stdout los contenidos de un archivo

- argc: imprime en stdout la cantidad de argumentos que recibió

- lineecho: recibe lineas en stdin y las reproduce en stdout

- gp: genera una exepción de General Protection

Estas aplicaciones demuestran todo el funcionamiento del núcleo, y dejan al
usuario en un sistema que simula las consolas modernas de hoy en día. Para
poder implementar estos programas fue necesario crear un archivo aparte,
apps/scall.asm, que es el puente entre las llamadas al sistema en C, que llaman
a una función con los parametros en la pila, y la verdadera llamada al sistema,
que pone los parametros en los registros de la CPU y genera la interrupción
0x80.

Luego el archivo apps/scall.asm es simplemente una colección de etiquetas de
todas las llamadas soportadas, que elige para cada una su respectivo número de
system call y lo coloca en eax. Luego se completan ebx, ecx y edx y se lanza la
interrupción.

En general todas las aplicaciones son bastante simples, como no les agregé
soporte a flags ni parametros raros por consola, solo realizan un trabajo
concreto (por ejemplo cp copia \_un\_ archivo de un lugar a otro). Entre todas
las aplicaciones, las más interesante es cash, en apps/cash.c.

Cash intenta de emular una consola moderna de un sistema tipo *nix. Presenta
información como el usuario y la carpeta actuales, y luego se queda
esperando a que se ingrese un comando. Luego pueden pasar una de dos cosas,
este comando puede ser manejado directamente por cash, o cash crea un nuevo
proceso con la aplicación deseada. La primera opción por ejemplo ocurre cuando
se ejecuta el comando "cd". Esto implica realizar la llamada al sistema chdir()
para cambiar el directorio actual, pero también requiere que se actualize el
texto del directorio actual que se le presenta al usuario (esto se encuentra en
la función updatepwd()). Todo este manejo de comandos es realizado en la
función execute().

La segunda opción es más simple, solo requiere que se ejecute la llamada
newprocess() con el programa y sus argumentos, y luego se bloquee esperando la
terminación del proceso hijo con waitpid().

Cash luego entra en un ciclo infinito en donde espera comandos del usuario, los
atiende y vuelve nuevamente al principio.

Como fue anticipado, todas las aplicaciones compilan sin modificaciones en
linux, ejecutando por ejemplo "gcc cp.c" para obtener cp. Hay 2 excepciones,
cash y ls. El problema con ls es que la llamada getdents() de linux no tiene un
wrapper en glibc (proveen readdir()), luego no puede ser llamada directamente
desde C de esta manera. La forma de solucionar esto es utilizar la función
syscall() con el número de getdents(), llamando indirectamente al kernel.

El problema con cash es de diseño, y ya fue explicado un poco anterioremente.
Para simplificar la implementación del kernel decidi reemplazar las llamadas
fork() y execv() por una unión de ambas, newprocess(). Por supuesto esto no
existe en linux ni en ningún sistema *nix, luego para que cash pueda ser
compilado por gcc para linux es necesario reemplazar esta llamada por un
fork() y seguido por un execv() inmediatamente, logrando el mismo resultado.
