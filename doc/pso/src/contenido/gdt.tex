\section{GDT}

Como se ha decidido utilizar un manejo de memoria plano, es decir evitar el uso
de segmentación en nuestro kernel (al igual que todo el resto de los sistemas
operativos modernos), la gdt solo cuenta con un número reducido de entradas.
Las definiciones de las mismas se pueden encontrar en kernel/gdt.c

Definimos 5 entradas: la primera nula según lo indican los manuales de Intel;
la segunda y tercera contienen los segmentos de código y datos del kernel
respectivamente, siendo las entradas 0x8 y 0x10 de la gdt; por último tenemos 2
entradas más similares a éstas, que solo modifican el campo dpl para ser de
usuario, siendo las 0x18 y 0x20 (aunque al cargarlas en los registros de
segmentos serán 0x1B y 0x23 para ajustar el RPL).

Todas estas entradas tienen como base el 0x0 y como límite el máximo posible,
ocupando de esta manera la totalidad de los 4GB direccionables.
