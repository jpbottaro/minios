\section{Manejo de memoria}

La implementación del manejo de memoria se encuentra en el archivo mm.c. Aquí
aparece por primera vez el uso de sys/queue.h, una implementación de listas
para C. Definí un arreglo conteniendo punteros a todas las páginas libres en el
sistema, ordenadas en una lista. De esta manera pedir páginas o devolverlas no
es más que quitarla o agregarla a dicha lista.

Por ésto la función mm\_init() inicializa la lista y le agrega todas las
páginas disponibles. Además registra la llamada al sistema palloc(), más de
esto más adelante.

La función mm\_mem\_alloc() simplemente toma la lista de páginas y recupera
el primer elemento. Luego procede a inicializarlo en 0 y devolverlo. Por
supuesto mm\_mem\_free() realiza lo contrario, vuelve a agregar la página en la
lista.

Tenemos las funciones mm\_dir\_new() y mm\_dir\_free() que crean y liberan
un directorio de páginas. Al crearlo, nos aseguramos realizar el identity
mapping de los primeros 4mb, donde reside el kernel, para que esté mapeado en
los directorios de páginas de todas las tareas. Liberarlo consiste en recorrer
el directorio y devolver las tablas de páginas, pero \_nada más\_. Esto se debe
a que por cuestiones de performance preferí utilizar una lista de páginas
usadas por proceso, para evitar tener que recorrer miles de tablas de páginas
con miles de entradas para liberar unas pocas.

Por último tenemos la llamada al sistema palloc, que la puse en el número 45
(vendría a reemplazar a brk). Nuevamente la implementación es muy sencilla,
toma una página de la lista de libres, la mapea en el directorio de páginas del
proceso que la pidió, y la devuelve.
