\section{Memoria Virtual}

Habiendo solucionado todas las perdidas de memoria el SO terminó siendo mucho
más estable y predecible. Pero la siguiente limitación que se presenta al querer
utilizarlo es el tope de la memoria. La solución clásica a este problema es
expandir la memoria con \textbf{memoria virtual}, utilizando otro medio de
almacenamiento como soporte.

Teniendo el driver de disco rígido completo para lectura/escritura, el siguiente
paso fue utilizarlo como extensión de la memoria principal.

\subsection{Módulo VMM}

El nuevo módulo de memoria virtual o vmm (virtual memory managment) es el
encargado de crear la ilusión de tener más memoria de la disponible físicamente.
La manera en que opera es muy simple: cuando la memoria principal se agota, se
\textbf{selecciona una página víctima} que será \textbf{desalojada}. Ésta página es
enviada al almacenamiento secundario (que puede ser cualquier dispositivo de
bloques, en nuestro caso el dísco rígido), y el marco que la contenía es
liberado.

En el caso de que algun proceso haga referencía a alguna página que no se
encuentre en memoria, el módulo va al disco a \textbf{recobrar la página}
desalojada y la deposita nuevamente en memoria (desalojando otra en caso
de ser necesario).

El módulo fue programado para que sea opcional. En caso de que se elimine la
llamada a $vmm\_init()$ de la inicialización del kernel, el sistema funcionará
perfectamente sin memoria virtual. Aún más, se podría habilitar en el medio de
la ejecución desde la consola misma de ser necesario (por ejemplo agregando una
syscall que llame a $vmm\_init()$)

Se agregó la aplicación 'eatmem' al sistema, con la que se pueden pedir
arbitrariamente páginas al sistema (que son escritas enseguida para evitar que
nos mienta el copy on write). De esta manera puede verse al módulo en acción,
cómo se acaba rápidamente la memoria en caso de deshabilitarlo, y la efectividad
de los distintos algorítmos de selección de víctimas.

Ver $mm\_mem\_alloc()$ y $pf\_handler$ en $kernel/mm.c$ para ver como es usado
el módulo vmm.

\subsection{Selección de víctimas}

Un paso muy importante en el proceso de memoria virtual es la selección de
víctimas, refiriendose a qué página desalojar en caso de ser necesario. Una mala
elección de algoritmo podría resultar en invertir la mayor parte del tiempo en
desalojar y recobrar páginas, problema conocido como trashing.

Otro punto a tener en cuenta es que algunas páginas son requeridas siempre y es
necesario (o conveniente) que nunca sean desalojadas. Denominados 'pinned' a las
páginas que preferimos nunca sean desalojadas, y proveemos una interfaz para
pedir páginas comunes o 'pinned' según sea necesario. Actualmente las únicas
páginas que siempre dejamos en memoria son las que conforman el directorio de
páginas, y las pilas de cada proceso.

Del conjunto de páginas 'non pinned', que incluyen todas las páginas de
código, datos y páginas pedidas dinámicamente, mantenemos una lista de posibles
víctimas.

Inicialmente se comenzó con un algoritmo FIFO. A lista de víctimas
siempre se le agregan elementos al final, y el selector siempre toma el primer
elemento. Si bien es un acercamiento muy siemple, se logran resultados
satisfactorios en muchos casos. El problema más grande de esta estrategia es que
no tiene en cuenta el uso de las páginas, por lo que termina desalojando páginas
importantes (como las de código del mismo proceso que está pidiendo más
memoria), que resultan en repetidos fallos de página que pueden ser evitados.

El mecanismo por excelencia de selección es el LRU. Si podemos calcular cuales
son las páginas más recientemente usadas (para alguna definicion de reciente y
de usada), podemos seleccionar víctimas más inteligentemente. El objetivo es
lograr ésto sin sacrificar demaciado tiempo calculando a quien desalojar.

Para esto aprovechamos el bit 'ACCESSED' de los directorios de páginas, que nos
indica si la página fue accedida desde la última vez que ese bit estuvo en 0. De
esta manera revisamos periodicamente este bit, y las páginas que se encuentren
marcadas son movidas al final de la cola de víctimas. Luego a la hora de elegir
la página a desalojar basta con tomar el primer elemento de dicha lista.

Esta estrategia es bastamente superior a FIFO, evitando tocar páginas
importantes como las de código en caso de que las mismas estén siendo
utilizadas.

Ver manejo de $victim\_list$ $kernel/mm.c$, y
$vmm\_select\_victim()$/$vmm\_lru\_watcher()$ en $kernel/vmm.c$. Para utilizar
FIFO basta con eliminar el watcher de reloj lru, que es registrado en
$vmm\_init()$

\subsection{Desalojar y recobrar página}

Una vez elegida la víctima, es necesario enviarla al dispositivo de bloques.
Mantenemos una estructura casi igual a la de la memoria principal para manejar
los marcos de página en el espacio de swap. Luego enviar la página solo implica
tomar un marco vacío y escribirlo.

El paso más complicado del desalojamiento es actualizar todas las referencias a
la página a desalojar. Ésto implica actualizar todos los directorios de páginas
de todos los procesos que pudiesen tener esta página, y marcarlos como no
presentes con una referencia a la página virtual en el nuevo almacenamiento.

El algoritmo actualemente toma la solución más simple ("when in doubt, use
bruteforce"). Recorre todas los directorios de páginas y actualiza sus
contenidos. Este mecanismo podría mejorarse manteniendo referencias en cada
página de los procesos que la usan, y evitar recorrer directorios de más.

Recobrar una página es similar. Cuando un proceso trata de acceder a una página
que desalojamos, se encontrará con que esta marcada como no presente, lo que
generará un fallo de página. El handler de esta interrupción usará el bit de
memoria virtual (un bit disponible en el directorio al que nosotros le dimos
esta semántica) para decidir si llamar al 'retriever' de páginas virtuales.

Así se lee de disco la página y se deposita en un marco libre de memoria,
actualizando nuevamente de la misma manera todas las referencias a dicha página
(sacando el bit de memoria virtual y levantando el de presente, con la dirección
física correspondiente).

Se tomó suma precaución para mantener actualizado el refcount de todas las
páginas en memoria y en almacenamiento secundario. De esta manera logramos
mantener únicamente las páginas en uso, evitando leaks de memoria.

Ver $vmm\_free\_page()$ y $vmm\_retrieve()$ en $kernel/vmm.c$.
