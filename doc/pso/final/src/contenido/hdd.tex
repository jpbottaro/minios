\section{Disco Rígido}

El driver de disco rígido existente fue diseñado originalmente
de manera simple para solo soportar lectura. Si bien se hizo un intento de
soportar escritura, nunca fue terminado. El primer paso consistió en
agregar \textbf{escritura para el hdd}.

El SO soporta el uso de device nodes (archivos que representan dispositivos en
el fs, ubicados en /dev), pero la manera en que estaba hecho era desordenada y
prensentaba problemas para habilitar el sistema booteando del hdd. El segundo
paso fue creado el módulo \textbf{devfs}.

Se aplicó una política más eficiente para el uso del hdd (o cualquiera sea el
dispositivo base) desde el sistema de archivos. Éste tercer paso involucró
mejorar el uso de \textbf{refcounts} y agregar la idea de \textbf{dirty} al FS.

Por último se arreglaron varios problemas generales, bugs lógicos en el código y
se reestructuraron funciones importantes.

\subsection{Escritura para el HDD}

Se mantuvo la decisión original de usar PIO para la lectura/escritura, lo que
significa que todas las transmisiones se realizaron a través de los puertos, sin
DMA.

Para lograr la escritura se adaptó parte del código responsable de leer del
dispositivo. La mayor parte del código es simple, se ubican los valores
requeridos en los puertos especificados y se lee/escribe lo necesario.

La parte más tumultosa de la implementación tiene que ver con la espera del
dispositivo. En varias ocasiones es necesario aplicar esperas forzosas de 400ms
para que el darle tiempo al disco de procesar los comandos. Ésto es
particularmente importante a la hora de escribir, y trajo muchos problemas hasta
que se descubrió que usar el comando de asm "rep" no era posible, ya que el
disco no lograba leer lo suficientemente rápido los datos ingresados. Agregando
un pequeño delay entre cada palabra, y esperando la bajada de la bandera BUSY
luego de cada sector, se logró soportar la escritura por puertos.

Ver $drivers/hdd$.

\subsection{DevFS}

Originalmente cada driver era el encargado de crear su propio device node en el
FS. Esto tenía dos problemas críticos: creaba una dependencia innecesaria de todos
los drivers con el FS, y resultaba en problemas a la hora de querer usar algún
dispositivo como base para el FS, creando una dependencia circular.

Por esta razón se creó una interfaz independiente encargada de crear device
nodes. De esta manera los drivers son inicializados, luego es el turno del
filesystem (tomando algún device previo como base), y por último devfs se
encarga de crear los device nodes correspondientes.

Esta estrategia es la utilizada originalmente por sistemas como Linux, tomando
responsabilidad de la creación de estos archivos virtuales. En la actualidad se
opta por una estrategia más modularizada, exponiendo información de dispositivos
y dejando que applicaciones de user-space manejen los device nodes (como es el
caso de udev). Actualmente esto simplemente lista los nodos a crear, se deja como
trabajo a futuro crear directorios como /sys y empujar código fuera del kernel
hacia aplicaciones comunes.

Ver $kernel/devfs.c$.

\subsection{Refcounts y Dirty en el FS}

La técnica original del sistema de archivos para grabar en el dispositivo donde
reside era la más simple, hacerlo siempre. Cada estructura del FS (esto incluye
a los inodos y los bloques) lleva su correspondiente refcount o contador de
referencias, y una vez que éste llega a 0 la estructura es liberada y grabada en
el dispositivo que soporta al FS.

La primer mejora consistió en revisar el mantenimiento de este refcount,
asegurandose de no perder referencias y dejar inodos/bloques sin liberar.

El cambio más importante fue el uso del flag "dirty". Siendo cuidadoso de marcar como
sucia toda estructura que reciba cambios en su ciclo de vida, es posible reducir
en un enorme grado la cantidad de escrituras al almacenamiento secundario.

Anteriormente leer un archivo en /a/b/c/hola.txt requería en
una escritura por cada lectura de inodo/bloque, en total 10 escrituras
aleatorias a disco (1 bloque y 1 inodo por archivo/carpeta, más el root).
El uso de dirty eliminó todas estas ineficiencias, haciendo que esta
operación no escriba nunca en disco.

Ver $fs/$ en general, en particular $inode.c$.
