El proyecto del nucleo que se nos presentó en la cursada contiene todo el código
y el build system en la misma carpeta. Se basa en un único makefile que se
encarga de compilar todo el código, crear la imagen y armar un diskette booteable.
Este diskette contiene un boot loader que simplemente carga toda la imagen en la
memoria y comienza la ejecución del mismo.

En un principio me propuse a armar un prototipo del FS directamente en la
carpeta de nuestro trabajo mini-kernel. Manteniendome fiel a la filosofía
Unix/C, comenzé a separar unidades lógicas en distintos archivos para organizar
el código.

El problema de esta organización es que no escala bien. Mientras más crecía el
código, mayores eran los archivos que se generaban y manejarlo no era práctico.
Por esto es que se me ocurrió jugar un poco con Make y generar un simple build
system para poder imitar el manejo de directorios de projectos como linux y
minix.

En resumen, organizé el codigo en los siguientes directorios:

- apps: algunas aplicaciones para utilizar en el entorno de Mini-Kernel 0.01

- bin: elementos para compilar/testear kernel

- doc: documentacion del trabajo

- fs: codigo del file system

- include: algunos archivos .h utiles

- kernel: el codigo del kernel con funcionalidad super-básica (incluye mmu,
interrupciones y hasta el scheduler)

- lib: funciones varias de uso en general (ej mystrncpy, mystrncmp, etc)

Cada directorio con algo de código contiene su propio makefile que se
encarga de compilar y armar todo lo que le corresponde. Luego en la carpeta
'bin' se encuentra el makefile central que se encarga de llamar a todos los
makefiles de las subcarpetas y linkear el kernel, finalizando con el armado del
diskette booteable.

Como el makefile central no sabe que es lo que hace cada sub-makefile (a
propósito), y más importante no sabe que archivos objeto crean, cada
sub-makefile se encarga de crear un archivo especial que especifica que es lo
que crea (que denomine 'depend'). El central simplemente lee este archivo por
cada sub-makefile y agrupa todos los archivos objeto para enviarselos al linker.

De esta manera agregar algún nuevo componente al sistema es mucho más cómodo,
solo es necesario avisarle al makefile central de la nueva carpeta, y luego
ocuparse de crear el sub-makefile únicamente para el nuevo código, sin tener que
modificar nada del resto del sistema. Es una manera modular de construir el
kernel, evitandose tener un mega-makefile complicado que este constantemente
cambiando, siendo difícil entenderlo y editarlo.

Como es costumbre en la mayoría de los proyectos de SO, se expone la carpeta
include/ con muchos de los headers de los distintos componentes del núcleo. En
particular se tiene include/minikernel/, que contiene las definiciones de
estructuras, signaturas y algunos macros que deben ser accesibles desde más de
1 componente (en nuestro caso es más bien para la interoperabilidad entre el
kernel y el fs).

Otro detalle del código es el idioma. En la actualidad el idioma predominante
para desarrollar es obviamente el inglés, y cualquier aplicación que uno quiere
compartir con el mundo por convensión se implementa en inglés, incluyendo nombre
de rutinas/variables/comentarios/etc. Es por ésto que antes de comenzar a
codificar nada nuevo me aseguré de modificar todo lo que se encuentra en
castellano al inglés, para mantener una consistencia con todo el projecto.

Por último, si el lector se encuentra interesado en el ambiente de programación
que se utilizo, la edición fue hecha exclusivamente con gvim/vim, la compilación
con nasm/gcc/ld, como se explicó recien todo se pega con make, y para testear el
sistema se utilizó la máquina virtual bochs. Si bien al principio el desarrollo
fue todo local, cuando la base del código ya estaba ordenada la manejé con git,
manteniendo una copia del repositorio en assembla.com (privado ya que el código
del kernel utilizado en la cursada no tiene licencia y no sabía si podía o no
publicarlo en una página pública como github, que maneja mucho mejor a git).
Para la programación del kernel en especial se hizo mucho uso de el programa
objdump, para decompilar los ejecutables y ver exactamente el código producido.

Para probar el kernel, simplemente ir al directorio 'bin', armar el diskette
booteable con 'make', y ejecutarlo en una máquina virtual con 'bochs'.
