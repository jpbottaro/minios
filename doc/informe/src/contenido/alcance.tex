\subsection{Kernel}

El kernel o nucleo con el que contamos es muy limitado, únicamente se encarga de
inicializar un sistema x86, setea ciertas estructuras necesarias para iniciar la
ejecución de aplicaciones, y finaliza su ejecución.

Las partes más importantes y desarrolladas de este kernel son el scheduler y la
atención de algunas interrupciones de usuario que conforman la interfaz del
núcleo con el resto del sistema.

Esta interfaz esta basada en el estandar de POSIX y implementa llamadas como
open/close/read/write etc., y de esta manera provee a las aplicaciones de
herramientas para crear nuevos procesos, leer y modificar archivos, terminar
la ejecución y demás. Estas rutinas estan implementadas en distintos lugares
del kernel, ya que cada una toca un sistema distinto. Por ejemplo open/close
se encuentran implementadas en el FS, mientras que la familia execve están en
el scheduler.

En cuanto a la implementación del scheduler, éste provee mecanismos para agregar
y remover procesos de la lista de ejecución, y se encarga de alternar los mismos
en el procesador para generar la ilución de multiprogramación. El algoritmo que
elegí para desarrollarlo fue el round-robin, que tiene como ventajas su simpleza
y escases de problemas graves como starvation.

Además, el nucleo cuenta con manejo simple del teclado y video para poder
interactuar con las aplicaciones en el sistema.

Por supuesto el sistema no cuenta con una biblioteca estandar de C
(implementaciones de la misma como glibc hacen \_extensivo\_ uso de todas las
llamadas de sistema disponibles en POSIX y todos sus flags/agregados/etc, por lo
que es imposible soportarla). Por lo tanto las aplicaciones apuntadas a este SO
deverán interactuar directamente con el nucleo mediante las llamadas al sistema.

\subsection{Minix FS}

La parte más importante del trabajo consiste en analizar e implementar el
sistema de archivos. Es importante aclarar que la imagen del sistema de archivos
se cargará completa en memoria al inicio del sistema, la razón siendo
simplemente que no buscamos estudiar ni desarrollar el manejo de disco, mucho
menos implementar toda la lógica de un disco IDE/ATA/SATA en un driver que puede
ocupar todo un libro.

Utilizando este camino, logramos no solo evitarnos problemas con disco y
drivers, sino que todos los cambios y modificaciones a la imagen no necesitan
ser pasadas por una cache, simplemente interactuamos con la imagen completa en
la memoria. Esto es lo que se logra en los SO modernos al utilizar un ramdisk
(aunque los fs en general desconocen esto y interactuan con el ramdisk como si
fuera un disco comun).

Con respecto a las rutinas de manejo de MinixFS, implementaré todo lo necesario
para interactuar normalmente con un sistema de archivos en un SO POSIX, es decir
open/close/read/write/lseek/unlink, incluyendo manejo de directorios con
chdir/mkdir/rmdir/getdents, entre otras, considerando las flags más importantes
de cada uno (por ejemplo O\_CREAT/O\_APPEND/O\_TRUNC en open).

Por último, hay 3 versiones de MinixFS, en este caso implementamos la versión 2.
Si bien comenzamos con su última versión, nos dimos cuenta que no existe mucho
soporte para la misma en linux (en especial mkfs.minix no sabe crear imagenes
versión 3) por lo que decidí adaptar las partes afectadas y remitirme a la
versión 2. De todas maneras la diferencia entre ambas es muy pequeña, solo se
aumenta el tamaño de algunos campos para soportar mayores discos, y se agregan
algunas utilidades no muy útiles para el trabajo.

\subsection{Consola y utilidades varias}

Todo el desarrollo descripto hasta ahora genera un SO primitivo que bootea la
computadora y provee funciones para manipularla. Para poder verlo en acción
introduzco algunas aplicaciones que manejan el sistema mediante llamadas al
sistema.

La aplicación más importante es la consola, es la única aplicación que es
llamada directamente desde el kernel, y es lo que marca la finalización del
booteo y el inicio de las aplicaciones de usuario. La consola permitirá recorrer
el sistema de archivos y a su vez lanzar otras aplicaciones.

También se proveen una batería de programas como ls, rm, cat, etc. que simulan a
los programas con los mismos nombres que se encuentran en sistemas *nix.
